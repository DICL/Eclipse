
\begin{abstract}

%In modern query processing systems, the caching facilities are distributed and
%scale in accordance with the number of servers. To maximize the overall system
%throughput, the distributed system should balance the query loads among
%servers \emph{and} also leverage cached results.  In particular, leveraging
%distributed cached data is becoming more important as many systems are being
%built by connecting many small heterogeneous machines rather than relying on a
%few high-performance workstations.  
%
%Although many query scheduling policies exist such as round-robin and
%load-monitoring, they are not sophisticated enough to both balance the load
%and leverage cached results.  
%

In this work, we present a novel distributed file processing framework called
\emph{UniDQP} that takes into account the dynamic nature of file I/O request 
distribution and remote contents of distributed caching infrastructure.  The
job scheduler of the framework makes scheduling decisions based on the 
spacial location of the queries. 
In addiction, we propose an innovate scheme enhancing the cohesion of system.
%I dont know how to describe the above sentence

%two layer back-end
%servers structure achieving a high cohesion with the scheduler.

\end{abstract}
